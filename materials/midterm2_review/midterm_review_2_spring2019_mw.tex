\documentclass[9pt]{beamer}
\usefonttheme[onlylarge]{structurebold}

%\documentclass[handout]{beamer}
%\usefonttheme[onlylarge]{structurebold}
%  \usepackage{pgfpages}
%\mode<handout>
%\pgfpagesuselayout{4 on 1}[letterpaper,border shrink=5mm]

\hypersetup{
  bookmarks = false,
  colorlinks,
  citecolor = red,
  linkcolor=blue,
  pdfpagemode=none,
  pdfstartview={Fit},
  pdftitle={},
  pdfauthor={Michael E. Waugh},
  pdfkeywords={} }
  \setbeamertemplate{navigation symbols}{}

\mode<presentation> {
  \usetheme{boxes}
  % or ...

  \setbeamercovered{transparent}
  % or whatever (possibly just delete it)
}

\setbeamertemplate{itemize subitem}[circle]
\setbeamerfont{frametitle}{size= \large}
\setbeamerfont{ framesubtitle }{size = \footnotesize}
\setbeamertemplate{frametitle}
{
\medskip
\smallskip
{\textsf{\underline{\insertframetitle\phantom{))))))))}}}}}


\usepackage[english]{babel}
\usepackage{wasysym}

\addfootbox{}{\hspace{6cm}\tiny {Midterm II Review---Economics of Global Business, Revised: \today}}%

\title[NYU Stern] % (optional, use only with long paper titles)
{\huge Solow Growth Model II}

\author[Michael Waugh] % (optional, use only with lots of authors)
{\bf{\Large}}%

\date[] % (optional)

\subject{Talks}

\begin{document}

%\begin{frame}
%  \titlepage
%\end{frame}


%%%%%%%%%%%%%%%%%%%%%%%%%%%%%%%%%%%%%%%%%%%%%%%%%%%%%%%%%%%%%%%%%%%%%%%%%%%%%%%%%%%%%%%%%%%%%%%%
%%%%%%%%%%%%%%%%%%%%%%%%%%%%%%%%%%%%%%%%%%%%%%%%%%%%%%%%%%%%%%%%%%%%%%%%%%%%%%%%%%%%%%%%%%%%%%%%

\begin{frame}[t]
\frametitle{Midterm Review: Big Picture}
\begin{itemize}
\item Stuff we learned
\begin{itemize}
\medskip
\item Ricardian model: autarky prices, prices at which trade occurs, real wages, pattern of specialization, pattern of trade, winners and losers, etc.
\medskip
\item Trade imbalances, connection with national savings and investment, open-economy loanable funds equilibrium. Connection with Chapter 3 concepts. 
\end{itemize}
\bigskip
\item Long-form questions questions will look like this:
\begin{itemize}
\medskip
\item Example: Policy ``x'' is this, what do you think?
\medskip
\item Could be a lot of answers. \begin{alertenv}{But the key to doing well is using tools/concepts we have discussed to answer the question}\end{alertenv}
\medskip
\item A good response: Policy ``x'' in the Ricardian model implies \ldots
\end{itemize}
\medskip
\item Format: 10 multiple choice. 2 long-form questions.
\end{itemize}
\bigskip
\end{frame}

%%%%%%%%%%%%%%%%%%%%%%%%%%%%%%%%%%%%%%%%%%%%%%%%%%%%%%%%%%%%%%%%%%%%%%%%%%%%%%%%%%%%%%%%%%%%%%%%
%%%%%%%%%%%%%%%%%%%%%%%%%%%%%%%%%%%%%%%%%%%%%%%%%%%%%%%%%%%%%%%%%%%%%%%%%%%%%%%%%%%%%%%%%%%%%%%%


\begin{frame}[t]
\frametitle{Cumulative Elements}
\begin{itemize}
\item Will be a ``TFP-like'' question on exam. Something changes, how does it affect production, factor prices, expenditure components.
\medskip
\item But will ask questions about how an open economy (NX $\neq$ 0) and how it things change. 
\medskip
\item If you score better on this relative to the ``TFP-like'' question on the first midterm, you get the Midterm 2 score. 
\end{itemize}
\end{frame}

%%%%%%%%%%%%%%%%%%%%%%%%%%%%%%%%%%%%%%%%%%%%%%%%%%%%%%%%%%%%%%%%%%%%%%%%%%%%%%%%%%%%%%%%%%%%%%%%
%%%%%%%%%%%%%%%%%%%%%%%%%%%%%%%%%%%%%%%%%%%%%%%%%%%%%%%%%%%%%%%%%%%%%%%%%%%%%%%%%%%%%%%%%%%%%%%%


\begin{frame}[t]
\frametitle{How to Prepare I}
\begin{itemize}
\item In class exercises, problem set, practice midterms, questions posed in slides.
\medskip
\item Slides
\medskip
\item Notes and Book for Trade Imbalances (Chapter 6-1 and 6-2).
\medskip
\item Blog
\medskip
\item See me if you need help
\end{itemize}
\end{frame}

%%%%%%%%%%%%%%%%%%%%%%%%%%%%%%%%%%%%%%%%%%%%%%%%%%%%%%%%%%%%%%%%%%%%%%%%%%%%%%%%%%%%%%%%%%%%%%%%
%%%%%%%%%%%%%%%%%%%%%%%%%%%%%%%%%%%%%%%%%%%%%%%%%%%%%%%%%%%%%%%%%%%%%%%%%%%%%%%%%%%%%%%%%%%%%%%%

\begin{frame}[t]
\frametitle{How to Prepare II}
Things to keep in mind
\medskip
\begin{itemize}
\item First, read through the whole exam. Where are all the points? How should I allocate
my time?
\medskip
\item Many times, the ``essay'' part of a question can be answered without having done
the computations. If you get stuck on a computation, don't give up on the whole
question.
\medskip
\item Before you begin an essay question, sketch the answer in the margin, or on a scrap
piece of paper. A few keywords in the order you would like to address them is usually
enough.\\
\smallskip
 \textbf{This may keep you from digressing, which wastes your time and waters down
your answer.}\\
\end{itemize}
\bigskip
\end{frame}

%%%%%%%%%%%%%%%%%%%%%%%%%%%%%%%%%%%%%%%%%%%%%%%%%%%%%%%%%%%%%%%%%%%%%%%%%%%%%%%%%%%%%%%%%%%%%%%%
%%%%%%%%%%%%%%%%%%%%%%%%%%%%%%%%%%%%%%%%%%%%%%%%%%%%%%%%%%%%%%%%%%%%%%%%%%%%%%%%%%%%%%%%%%%%%%%%

\begin{frame}[t]
\frametitle{Details}
\bigskip
\begin{itemize}
\item Start $\approx$ 3:30, so be on time!
\bigskip
\item It will take 75 minutes.
\bigskip
\item You can use one sheet of notes: letter paper, both sides, any size type you like.
\bigskip
\item You may also use a calculator, but may not use any device capable of wireless transmission. Proximity to any such device during the exam will be treated as a violation of the honor
code.
\bigskip
\end{itemize}
\bigskip
\end{frame}



\end{document} 