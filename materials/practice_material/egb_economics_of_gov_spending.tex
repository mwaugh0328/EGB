\documentclass[12pt,pdftex,twoside,letterpaper]{exam}
\usepackage{amsmath,amssymb, amsthm}
\usepackage{graphicx}
\usepackage{color}
\usepackage{comment}
\usepackage{layout}
\usepackage{booktabs}
\usepackage[flushleft]{threeparttable}
\usepackage{caption}
\usepackage{setspace}
\usepackage{float}
\usepackage[colorlinks=true,linkcolor=blue,citecolor=black,urlcolor=blue,bookmarks=false,pdfstartview={FitV}]{hyperref}

%%%%%%%%%%%%%%%%%%%%%%Margins%%%%%%%%%%%%%%%%%%%%%%%%%%%%%%%%%%%%%%%%%%%%%%%
\usepackage[margin=1.0in]{geometry}
\setlength{\parindent}{0in}
\setlength{\parskip}{.09in}
\raggedbottom

%%%%%%%%%%%%%%%%%%%%Tighten up the lists%%%%%%%%%%%%%%%%%%%%%%%%%%%%%%%%%%
\let\OLDdescription\description
\renewcommand\description{\OLDdescription\setlength{\itemsep}{-2mm}}

%%%%%%%%%%%%%%%%%%%%%%Headers and Footers%%%%%%%%%%%%%%%%%%%%%%%%%%%%%%%%%%
\pagestyle{headandfoot}
\runningheadrule
\firstpageheadrule
\firstpageheader{\includegraphics[width=0.25\textwidth]{../syllabus/stern_black1.pdf}}{}{Economics of Global Business}
\runningheader{Economics of Global Business}{}{Syllabus: ECON-UB.0011.02}
\runningfooter{}{}{}

%%%%%%%%%%%%%%%%%%%%%%%%%Exam class formating%%%%%%%%%%%%%%%%%%%%%%%%%%%%%%%%%%%%%%%%
%\printanswers
\renewcommand{\partlabel}{\thepartno.}
\renewcommand{\questionshook}{\setlength{\itemsep}{0.2in}}
\renewcommand{\partshook}{\setlength{\leftmargin}{0.2in}}
\renewcommand\familydefault{\sfdefault}

%%%%%%%%%%%%%%%%%%%%%%%%%%%%%%%%%Let Backus fight the good fight, I'm going with LN%%%%%%%%%%%%%%%%%%%
\renewcommand{\log}{\ln}

\begin{document}
\centerline{\large\bf Practice: The economics of Government Spending (AOC, GND, GDP)}
\vspace{1mm}
\centerline{\large\bf Spring 2019}
\vspace{3mm}
\centerline{\bf Revised:  \today}
\bigskip
\bigskip
%\vspace{-.25cm}

\textbf{Step 1.} First ask your self the following question: Can government spending increase GDP? Why or why not?\\
\bigskip



\textbf{Step 2.} AOC has advocated the ``Green New Deal'' henceforth GND.

I don't know what is in it, but lets suppose that the GND is simply an increase in government expenditure. \textbf{$G$ goes up|that's it. No change in $L$, $K$, or tax rates, TFP, etc...}
\begin{itemize}
\item Does this plan change the number of people employed or quantity of capital?

\item Does this plan change GDP?
\end{itemize}
\bigskip

\textbf{Step 3.} Ok now reflect on these questions...
\begin{itemize}
\item Does this plan change national savings? What is the issue?

\item How would this plan change the quantity of investment in the economy? Real interest rates?
\end{itemize}
\bigskip

\textbf{Step 4.} AOC decides to change the plan and finance the GND with a corresponding increase in taxes. \textbf{$T$ goes up by the exact same amount as $G$.} The plan is what experts would call ``deficit neutral.''
\begin{itemize}
\item Does this plan change national savings? What is the issue?

\item How would this plan change the quantity of investment in the economy?
\end{itemize}
\bigskip


\end{document} 