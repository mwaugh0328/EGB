\documentclass[9pt]{beamer}
\usefonttheme[onlylarge]{structurebold}

%\documentclass[handout]{beamer}
%\usefonttheme[onlylarge]{structurebold}
%  \usepackage{pgfpages}
%\mode<handout>
%\pgfpagesuselayout{4 on 1}[letterpaper,border shrink=5mm]

\hypersetup{
  bookmarks = false,
  colorlinks,
  citecolor = red,
  linkcolor=blue,
  pdfpagemode=none,
  pdfstartview={Fit},
  pdftitle={},
  pdfauthor={Michael E. Waugh},
  pdfkeywords={} }
  \setbeamertemplate{navigation symbols}{}

\mode<presentation> {
  \usetheme{boxes}
  % or ...

  \setbeamercovered{transparent}
  % or whatever (possibly just delete it)
}

\setbeamertemplate{itemize subitem}[circle]
\setbeamerfont{frametitle}{size= \large}
\setbeamerfont{ framesubtitle }{size = \footnotesize}
\setbeamertemplate{frametitle}
{
\medskip
\smallskip
{\textsf{\underline{\insertframetitle\phantom{))))))))}}}}}


\usepackage[english]{babel}
\usepackage{wasysym}

\addfootbox{}{\hspace{6cm}\tiny {Midterm I Review---Economics of Global Business, Revised: \today}}%

\title[NYU Stern] % (optional, use only with long paper titles)
{\huge Solow Growth Model II}

\author[Michael Waugh] % (optional, use only with lots of authors)
{\bf{\Large}}%

\date[] % (optional)

\subject{Talks}

\begin{document}

%\begin{frame}
%  \titlepage
%\end{frame}


%%%%%%%%%%%%%%%%%%%%%%%%%%%%%%%%%%%%%%%%%%%%%%%%%%%%%%%%%%%%%%%%%%%%%%%%%%%%%%%%%%%%%%%%%%%%%%%%
%%%%%%%%%%%%%%%%%%%%%%%%%%%%%%%%%%%%%%%%%%%%%%%%%%%%%%%%%%%%%%%%%%%%%%%%%%%%%%%%%%%%%%%%%%%%%%%%

\begin{frame}[t]
\frametitle{Midterm Review: (Only for Monday/Wednesday Class)}
\begin{itemize}
\item Broad overview...
\begin{itemize}
\medskip
\item Measuring GDP, measuring labor market performance, inflation.
\medskip
\item Production function.
\medskip
\item Marginal products and the distribution of income.
\medskip
\item Expenditure side of GDP.
\medskip
\item Connect investment today with capital in future.
\end{itemize}
\bigskip
\item What this will look like...
\begin{itemize}
\medskip
\item Example: Policy ``x'' is this, what do you think?
\medskip
\item Could be a lot of answers. \begin{alertenv}{But the key to doing well is using tools/concepts we have discussed to answer the question}\end{alertenv}
\medskip
\item A good response: Policy ``x'' in the BGP model implies \ldots
\end{itemize}
\medskip
\item Format: 10 multiple choice. 2 long-form questions. Just like practice midterms.
\end{itemize}
\bigskip
\end{frame}

%%%%%%%%%%%%%%%%%%%%%%%%%%%%%%%%%%%%%%%%%%%%%%%%%%%%%%%%%%%%%%%%%%%%%%%%%%%%%%%%%%%%%%%%%%%%%%%%
%%%%%%%%%%%%%%%%%%%%%%%%%%%%%%%%%%%%%%%%%%%%%%%%%%%%%%%%%%%%%%%%%%%%%%%%%%%%%%%%%%%%%%%%%%%%%%%%


\begin{frame}[t]
\frametitle{How to Prepare I}
\begin{itemize}
\item In class exercises, problem sets, practice midterm, questions posed in slides, practice problems in back of chapters in Mankiw.
\medskip
\item Slides
\medskip
\item Mankiw (Chapter 2 and Chapter 3).
\medskip
\item Blog
\end{itemize}
\end{frame}

%%%%%%%%%%%%%%%%%%%%%%%%%%%%%%%%%%%%%%%%%%%%%%%%%%%%%%%%%%%%%%%%%%%%%%%%%%%%%%%%%%%%%%%%%%%%%%%%
%%%%%%%%%%%%%%%%%%%%%%%%%%%%%%%%%%%%%%%%%%%%%%%%%%%%%%%%%%%%%%%%%%%%%%%%%%%%%%%%%%%%%%%%%%%%%%%%

\begin{frame}[t]
\frametitle{How to Prepare II}
Things to keep in mind
\medskip
\begin{itemize}
\item First, look through the whole exam: \\
\medskip
\textbf{Think\ldots How should I allocate my time?}
\medskip
\item Many times, parts of a question can be answered without having done
the other parts. If you get stuck on a computation, don't give up on the whole
question.
\medskip
\item Before you begin an essay question, sketch the answer in the margin, or on a scrap
piece of paper. A few key concepts in the order you would like to address them is usually
enough.\\
\medskip
 \textbf{This will keep you from digressing, which wastes your time and waters down
your answer.}\\
\end{itemize}
\bigskip
\end{frame}

%%%%%%%%%%%%%%%%%%%%%%%%%%%%%%%%%%%%%%%%%%%%%%%%%%%%%%%%%%%%%%%%%%%%%%%%%%%%%%%%%%%%%%%%%%%%%%%%
%%%%%%%%%%%%%%%%%%%%%%%%%%%%%%%%%%%%%%%%%%%%%%%%%%%%%%%%%%%%%%%%%%%%%%%%%%%%%%%%%%%%%%%%%%%%%%%%

\begin{frame}[t]
\frametitle{Details}
\bigskip
\begin{itemize}
\item Start $\approx$ 3:30, so be on time!
\bigskip
\item Allotted 75 minutes to complete.
\bigskip
\item You can use one sheet of notes: letter paper, both sides, any size type you like.
\bigskip
\item You may also use a calculator, but may not use any device capable of wireless transmission. Proximity to any such device during the exam will be treated as a violation of the honor
code.
\bigskip
\end{itemize}
\bigskip
\end{frame}



\end{document} 