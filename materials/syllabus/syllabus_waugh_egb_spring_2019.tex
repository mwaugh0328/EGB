\documentclass[12pt,pdftex,twoside,letterpaper]{exam}
\usepackage{amsmath,amssymb, amsthm}
\usepackage{graphicx}
\usepackage{color}
\usepackage{comment}
\usepackage{layout}
\usepackage{booktabs}
\usepackage[flushleft]{threeparttable}
\usepackage{caption}
\usepackage{setspace}
\usepackage{float}
\usepackage[colorlinks=true,linkcolor=blue,citecolor=black,urlcolor=blue,bookmarks=false,pdfstartview={FitV}]{hyperref}

%%%%%%%%%%%%%%%%%%%%%%Margins%%%%%%%%%%%%%%%%%%%%%%%%%%%%%%%%%%%%%%%%%%%%%%%
\usepackage[margin=1.0in]{geometry}
\setlength{\parindent}{0in}
\setlength{\parskip}{.09in}
\raggedbottom

%%%%%%%%%%%%%%%%%%%%Tighten up the lists%%%%%%%%%%%%%%%%%%%%%%%%%%%%%%%%%%
\let\OLDdescription\description
\renewcommand\description{\OLDdescription\setlength{\itemsep}{-2mm}}

%%%%%%%%%%%%%%%%%%%%%%Headers and Footers%%%%%%%%%%%%%%%%%%%%%%%%%%%%%%%%%%
\pagestyle{headandfoot}
\runningheadrule
\firstpageheadrule
\firstpageheader{\includegraphics[width=0.25\textwidth]{stern_black1.pdf}}{}{Economics of Global Business}
\runningheader{Economics of Global Business}{}{Syllabus: ECON-UB.0011.02}
\runningfooter{}{}{}

%%%%%%%%%%%%%%%%%%%%%%%%%Exam class formating%%%%%%%%%%%%%%%%%%%%%%%%%%%%%%%%%%%%%%%%
%\printanswers
\renewcommand{\partlabel}{\thepartno.}
\renewcommand{\questionshook}{\setlength{\itemsep}{0.2in}}
\renewcommand{\partshook}{\setlength{\leftmargin}{0.2in}}
\renewcommand\familydefault{\sfdefault}

%%%%%%%%%%%%%%%%%%%%%%%%%%%%%%%%%Let Backus fight the good fight, I'm going with LN%%%%%%%%%%%%%%%%%%%
\renewcommand{\log}{\ln}

\begin{document}
\centerline{\Large\bf Syllabus:  Economics of Global Business}
\vspace{1mm}
\centerline{\large\bf ECON-UB.0011 $|$ Spring 2019}
\vspace{3mm}
\centerline{\bf Revised:  \today}

\vspace{-.25cm}

\subsubsection*{About the Course}

The Economics of Global Business is an intermediate level course in macroeconomics, with particular attention on international economics. The goal of this course is to provide a coherent framework for analyzing international macroeconomic events (e.g. long-run economic growth, the rise of globalization, recessions, and booms) and macroeconomic policies (e.g. monetary, fiscal, trade, and currency policy).

\subsubsection*{About the Instructor}
Professor Michael Waugh\\
Email: mwaugh@stern.nyu.edu\\
Office: KMC 7-74\\
Phone: 212-998-0288\\
Office hours: TBA.

\subsubsection*{Where and When}
\begin{itemize}
\item Section .01: TISCH UC24, Tuesday and Thursday 11:00 AM - 12:15 PM.
\item Section .03: KMEC 4-80, Monday and Wednesday 3:30 PM - 4:45 PM.
\end{itemize}

\subsubsection*{Prerequisites}

Course pre-requisite: ECON-UB.0001 Microeconomics.

\subsubsection*{Course Material}
\begin{itemize}
\item \textbf{Course Website.} All course materials are posted here:
 \url{https://mwaugh0328.github.io/EGB/}. Bookmark this site. Check it regularly.
 I will not be using NYU Classes (except for grades).


\item \textbf{Slides.}
Most of the lectures will be accompanied with slides. These provide access to the core of the material we will cover. Versions of these slides will be distributed in class.

\item \textbf{Textbook.}
There is one required text book It is \href{https://www.amazon.com/Macroeconomics-N-Gregory-Mankiw/dp/1464182892}{Macroeconomics} by N. Gregory Mankiw. The 9th edition is the most recent version, but previous editions are cheaper and work just as well. These books should be available for purchase at the NYU Bookstore and other outlets (e.g. Amazon).

I have also created eBook option. The eBook is somewhat cheaper and only includes the chapters that we will cover during this course. To purchase follow the link below:
\begin{itemize}
\item \url{http://www.macmillanhighered.com/launchpad/mankiw9e/3004320}
\item At this link should be a website specifying that this is the book associated with my class. From here you can purchase access to the ebook. After you have purchase the book, you can use this same link to access the book.

\item Note that only the ebook will be hosted on this website. All other course material is posted here:
 \url{https://mwaugh0328.github.io/EGB/}
\end{itemize}

\item \textbf{My Blog for the Course.} I operate a blog for the course. This is an area for me to (1) communicate with you about current events/economic news and how they fit in with the class material and (2) discuss more nuanced issues related to class material. You should follow this forum regularly.

\item \textbf{Lecture notes.}
For the modules focusing on international issues (trade and finance), I will provide notes written by myself. Hard copies of these notes will be issued to you a week or so before they are required and they be posted online as well.

\item \textbf{Economic News.}
Aside from the regular course material, students are expected to follow current economic events. While no subscription to any particular newspaper or journal is required, you should regularly read any of the following: The FT, Wall Street Journal, The Economist.
\end{itemize}


\subsubsection*{Help}

There are times when everyone needs a little help. If that happens to you, please let me know. My office hours are before/after class, but I'm around most of the time and would be happy to talk if not otherwise engaged. I'm in KMC 7-74.

During day time on weekdays, I generally answer email (\href{mailto:mwaugh@stern.nyu.edu}{mwaugh@stern.nyu.edu}) quickly. During evenings and weekends, you should expect that my response will be less prompt. My official policy is that I reserve the right to get back to you within 24 hours. If I don't, feel free to email me again.

There is one teaching fellow associated with the course. He will hold regular office hours during the week.\\

\textbf{Teaching Fellows}\\
\\
TBA

\newpage

\subsubsection*{Grades}

Your grade will be based on:
\begin{center}
\begin{tabular}{lcc}
    Professionalism &\hspace*{0.50in}&   5\%  \\
    Online Quizzes && 10\% \\
    Problem Sets &&  10\% \\
    Midterm \# 1 &&  20\% \\
    Midterm \# 2 &&  20\% \\
    Final &&  35\% \\
\end{tabular}
\end{center}
\begin{itemize}
\item \textbf{Class attendance, participation, professionalism}. Professionals show up and are prepared---this is my expectation of you. You should come to class prepared to discuss assigned topics and current issues of the day. Your thoughtful participation makes the course more interesting and productive for everyone, including me. You can excel in this area if you come to class on time and contribute to the course by:
\begin{itemize}
\item Listening attentively in class and volunteering to address my review questions.
\item Providing evidence of having reviewed what done in previous classes.
\item Demonstrating interest in your peers' comments and questions.
\item Advancing the discussion by contributing insightful comments and questions.
\item Following the guide for professional behavior described below.
\end{itemize}

\item \textbf{Online Quizzes}. On the course website, I will post online quizzes (via Google Forms) to be taken. They will be short multiple choice type quizzes, they will occur at about a week and a half intervals, and you will have multiple days to complete them at your leisure. \textbf{If the online quizzes are not completed within the given time period, then you will receive a zero and there are no make up opportunities.}

\item \textbf{Problem Sets}. In the problem sets you will be asked to compute numerical examples similar to ones we have covered in class, find and analyze data, and use the principles we are studying to analyze and comment on various issues.

    For problem sets, students may work in groups (no larger than three students). All students in the group receive the same grade.

    Problem sets are graded as ``check,'' ``check plus,'' or ``check minus.'' Problem sets scored check or check plus earn full credit. Problem sets graded check minus earn zero credit. \textbf{Late problem sets are not accepted.}

\item \textbf{Exams}. There will be two in-class midterms. The midterms will be on:
\begin{itemize}
\item\textbf{Midterm \#1: Section 3 on Wednesday 3/13; Section 1 on Thursday 3/14}
\item\textbf{Midterm \#2: Section 3 on Wednesday 4/17; Section 1 on Thursday 4/18}
\end{itemize}
The Final will take place during the \href{http://www.nyu.edu/registrar/pdf/Final_exam_schedule_Spring_2019.pdf}{University specified time}: \textbf{All sections, 8:00am, May 21, 2019}.

In these exams, you may use one sheet of notes: letter paper, both sides, any size type you like. You may also use a calculator, but may not use any device capable of wireless transmission.  Proximity to any such device during the exam will be treated as a violation of the honor code (see below).

\item \textbf{Examination-Make-up Test Policy:} \textbf{THERE ARE NO MAKE UPS.}\\

If you miss the midterm for a justified reason and provide sufficient evidence, your final score will be counted instead of the midterm exam. If you miss the midterm for any other reason you will receive a zero. If for a justified reason you miss the final exam, you will receive an incomplete, which has to be removed in the earliest possible semester. My discretion determines what a ``justified reason'' is.

\item \textbf{Re-grading requests.} The process of assigning grades is time consuming and much effort is put into the grading of exams to yield scores that are an unbiased evaluation of your performance.

    \textbf{Students are encouraged to respect the integrity and authority of my grading system and are discouraged from pursuing arbitrary challenges to it.}

If you believe an inadvertent error has been made in grading, a request to have the grade re-evaluated may be submitted. You must submit such requests in writing to me within 7 days of receiving the grade, including a clear statement as to why you believe that an error in grading has been made. There will be an automatic regrade of the entire exam.

\end{itemize}

\subsubsection*{Course Website}
First, \textbf{I will not be using NYU Classes}. Everything you need, including this document, is posted on
the {\bf course website\/}:
%
\vspace{-0.15in}
\begin{center}
\url{https://mwaugh0328.github.io/EGB/}
\end{center}

Second, virtually everything you need for this course will be posted on the course website: notes, assignments, slides, and links to information sources. Some online documents, including this one, have links to outside sources that may not be apparent in the printed version.

\subsubsection*{Professional Behavior}
In the interest of having a high-quality experience for all,
your classmates and I ask that you:
\begin{itemize}
\item Attend class.

\item Arrive to class on time and stay to the end of the class period. Chronically arriving late or leaving class early is unprofessional and disruptive to the entire class.

\item Put away your laptop and Iphone/Ipad/mobile phone, etc. I strongly prefer that you do not use your laptops in class as I receive many complaints every year from other students that find such laptop use disruptive.

\item Eating or drinking in class in any way that interferes with class decorum is inappropriate.

\item Be respectful to myself, the teaching fellow, and your classmates.

\item Join your classmates. Avoid the last row. Be Engaged!
\end{itemize}

\subsubsection*{Grades Distribution}
At NYU Stern we seek to teach challenging courses that allow students to demonstrate their mastery of the subject matter.  In general, students in undergraduate core courses can expect a grading distribution where:
\begin{itemize}
\item 25-35\% of students can expect to receive A's for excellent work
\item 50-60\% of students can expect to receive B's for good or very good work
\item 5-10\% of students can expect to receive C's or less for adequate or below work
\end{itemize}
Note that while the School uses these ranges as a guide, the actual distribution for this course and your own grade will depend upon how well you actually perform in this course.



\subsubsection*{Stern Policies}
\begin{itemize}
\item \textbf{General Behavior.} The School expects that students will conduct themselves with respect and professionalism toward faculty, students, and others present in class and will follow the rules laid down by the instructor for classroom behavior.  Students who fail to do so may be asked to leave the classroom.

\item \textbf{Collaboration on Graded Assignments.} Students may not work together on any assignments or exams unless the instructor gives express permission.

\item \textbf{Course Evaluations.} Course evaluations are important to us and to students who come after you.  Please complete them thoughtfully.

\item \textbf{Academic Integrity.}
\begin{itemize}
\item Integrity is critical to the learning process and to all that we do here at NYU Stern. As members of our community, all students agree to abide by the NYU Stern Student Code of Conduct, which includes a commitment to:
\item Exercise integrity in all aspects of one's academic work including, but not limited to, the preparation and completion of exams, papers and all other course requirements by not engaging in any method or means that provides an unfair advantage.
\item Clearly acknowledge the work and efforts of others when submitting written work as one's own. Ideas, data, direct quotations (which should be designated with quotation marks), paraphrasing, creative expression, or any other incorporation of the work of others should be fully referenced.
\item Refrain from behaving in ways that knowingly support, assist, or in any way attempt to enable another person to engage in any violation of the Code of Conduct. Our support also includes reporting any observed violations of this Code of Conduct or other School and University policies that are deemed to adversely affect the NYU Stern community.
\end{itemize}
The entire Stern Student \href{http://www.stern.nyu.edu/sites/default/files/assets/documents/con_039512.pdf}{Code of Conduct} applies to all students enrolled in Stern courses. Discussion of this code of conduct and its application to the undergraduate college is here: \href{http://www.stern.nyu.edu/uc/codeofconduct}{Undergraduate College Conduct}. \textbf{Any violation of the policies pertaining to Academic Integrity will result in a failing grade for the course.}
\end{itemize}

\subsubsection*{Students with disabilities}

If you have a qualified disability that requires academic accommodation during this course,
please contact the Moses Center for Students with Disabilities (CSD, 212-998-4980) and ask them to
send me a letter verifying your registration and outlining the accommodation they recommend.
If you need to take an exam at the CSD, you must submit a completed Exam Accommodations Form to them
at least one week prior to the scheduled exam time to be guaranteed accommodation.

\vfill \centerline{\it \copyright \ \number\year \
NYU Stern School of Business}


\end{document}
