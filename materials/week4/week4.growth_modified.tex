%\documentclass[10pt]{beamer}
%\usefonttheme[onlylarge]{structurebold}

\documentclass[handout]{beamer}
\usefonttheme[onlylarge]{structurebold}
  \usepackage{pgfpages}
\mode<handout>
\pgfpagesuselayout{4 on 1}[letterpaper,border shrink=5mm]

\hypersetup{
  bookmarks = false,
  colorlinks,
  citecolor = red,
  linkcolor=blue,
  pdfpagemode=none,
  pdfstartview={Fit},
  pdftitle={},
  pdfauthor={Michael E. Waugh},
  pdfkeywords={} }
  \setbeamertemplate{navigation symbols}{}

\mode<presentation> {
  \usetheme{boxes}
  % or ...

  \setbeamercovered{transparent}
  % or whatever (possibly just delete it)
}

\setbeamertemplate{itemize subitem}[circle]
\setbeamerfont{frametitle}{size= \large}
\setbeamerfont{ framesubtitle }{size = \footnotesize}
\setbeamertemplate{frametitle}
{
\medskip
\smallskip
{\textsf{\underline{\insertframetitle\phantom{))))))))}}}}}


\usepackage[english]{babel}
\usepackage{wasysym}

\addfootbox{}{\hspace{2cm}\tiny {Accounting For Growth---Economics of Global Business, Revised: \today}}%

\title[NYU Stern] % (optional, use only with long paper titles)
{\Large Accounting For Growth}

\author[Michael Waugh] % (optional, use only with lots of authors)
{\bf{\Large}}%

\date[] % (optional)

\subject{Talks}

\begin{document}

\begin{frame}
  \titlepage
\end{frame}


%%%%%%%%%%%%%%%%%%%%%%%%%%%%%%%%%%%%%%%%%%%%%%%%%%%%%%%%%%%%%%%%%%%%%%%%%%%%%%%%%%%%%%%%%%%%%%%%
%%%%%%%%%%%%%%%%%%%%%%%%%%%%%%%%%%%%%%%%%%%%%%%%%%%%%%%%%%%%%%%%%%%%%%%%%%%%%%%%%%%%%%%%%%%%%%%%

\begin{frame}[t]
\frametitle{One Question\ldots}
\bigskip
\begin{itemize}
\medskip
\item Where does economic growth come from?
\begin{itemize}
\medskip
\item Link our model from Chapter 3 with how capital changes over time.
\medskip
\item An approach to determine the sources of economic growth (Appendix Chapter 9; PS \#2).
\end{itemize}
\end{itemize}
\end{frame}


%%%%%%%%%%%%%%%%%%%%%%%%%%%%%%%%%%%%%%%%%%%%%%%%%%%%%%%%%%%%%%%%%%%%%%%%%%%%%%%%%%%%%%%%%%%%%%%%
%%%%%%%%%%%%%%%%%%%%%%%%%%%%%%%%%%%%%%%%%%%%%%%%%%%%%%%%%%%%%%%%%%%%%%%%%%%%%%%%%%%%%%%%%%%%%%%%

\begin{frame}[t]
\frametitle{Capital Accumulation}
\footnotesize
\begin{itemize}
\item Capital accumulation equation
\begin{eqnarray*}
K_{t+1} = (1 - \delta) K_{t} + I_t
\end{eqnarray*}
\medskip
\item $\delta$ is the rate of depreciation of capital.
\bigskip
\item Savings equals investment (loanable funds equilibrium) (end of Chapter 3)!
\begin{eqnarray*}
I_t = S_t = (Y_t - T_t) - \beta \times(Y_t- T_t) + (T_t - G_t)
\end{eqnarray*}
\begin{enumerate}
\footnotesize
\smallskip
\item Changes that affect production \textbf{today} ($A$, $L$, or $K$) affect investment and in turn capital tomorrow!
\medskip
\medskip
\item Changes that affect spending behavior \textbf{today} ($\beta$ or $T$ or $G$) affect investment and in turn capital tomorrow!
\end{enumerate}
\end{itemize}
\end{frame}

%%%%%%%%%%%%%%%%%%%%%%%%%%%%%%%%%%%%%%%%%%%%%%%%%%%%%%%%%%%%%%%%%%%%%%%%%%%%%%%%%%%%%%%%%%%%%%%%
%%%%%%%%%%%%%%%%%%%%%%%%%%%%%%%%%%%%%%%%%%%%%%%%%%%%%%%%%%%%%%%%%%%%%%%%%%%%%%%%%%%%%%%%%%%%%%%%

\begin{frame}[t]
\frametitle{Important food for thought\ldots}
\footnotesize
\begin{itemize}
\item Previous equation now links stuff happening today with stuff in the future through capital accumulation.
\medskip
\item Now we can distinguish between short run ($K$ is fixed) and then long run (as $K$ changes).
\medskip
\item Return to our two examples:
\begin{enumerate}
\footnotesize
\smallskip
\item \textbf{The Wall}... how would you expect $K$ to change? What will happen to GDP and wages, real interest rates as this takes place?
\medskip
\medskip
\item \textbf{The GND}... how would you expect $K$ to change? What will happen to GDP and wages, real interest rates as this takes place?
\end{enumerate}
\end{itemize}
\end{frame}


%%%%%%%%%%%%%%%%%%%%%%%%%%%%%%%%%%%%%%%%%%%%%%%%%%%%%%%%%%%%%%%%%%%%%%%%%%%%%%%%%%%%%%%%%%%%%%%%
%%%%%%%%%%%%%%%%%%%%%%%%%%%%%%%%%%%%%%%%%%%%%%%%%%%%%%%%%%%%%%%%%%%%%%%%%%%%%%%%%%%%%%%%%%%%%%%%

\begin{frame}[t]
\frametitle{Accounting For Growth}
\begin{itemize}
\item Our production function\ldots
\begin{eqnarray*}
Y &= A \times F(K,L)\\
\\
 &=  A K^{\alpha}L^{1-\alpha}
\end{eqnarray*}
\item Only three ways output can grow\ldots
\begin{itemize}
\bigskip
\item More capital ($K$), aka ``capital deepening''
\bigskip
\item More labor inputs ($L$), may be from more workers, better educated workers, etc. aka ``labor deepening''
\bigskip
\item Better TFP ($A$), aka ``technological progress''
\end{itemize}
\end{itemize}
\end{frame}

%%%%%%%%%%%%%%%%%%%%%%%%%%%%%%%%%%%%%%%%%%%%%%%%%%%%%%%%%%%%%%%%%%%%%%%%%%%%%%%%%%%%%%%%%%%%%%%%
%%%%%%%%%%%%%%%%%%%%%%%%%%%%%%%%%%%%%%%%%%%%%%%%%%%%%%%%%%%%%%%%%%%%%%%%%%%%%%%%%%%%%%%%%%%%%%%%
\begin{frame}[t]
\frametitle{Step \#1: Measure $K$ and $L$}
\begin{itemize}
\item $K$ is measured as the accumulation of investments (PPE) over time (see previous slides).
\bigskip
\item $L$ is measured in various ways
\begin{itemize}
\medskip
\item Number of people working, e.g. labor force participation multiplied by the population.
\medskip
\item Adjusted for how hard they work, e.g. hours worked per person, etc.
\medskip
\item Adjusted for how well they work, e.g. education adjusted, etc.
\end{itemize}
\end{itemize}
\end{frame}

%%%%%%%%%%%%%%%%%%%%%%%%%%%%%%%%%%%%%%%%%%%%%%%%%%%%%%%%%%%%%%%%%%%%%%%%%%%%%%%%%%%%%%%%%%%%%%%%
%%%%%%%%%%%%%%%%%%%%%%%%%%%%%%%%%%%%%%%%%%%%%%%%%%%%%%%%%%%%%%%%%%%%%%%%%%%%%%%%%%%%%%%%%%%%%%%%

\begin{frame}[t]
\frametitle{Step \#2: Measure TFP}
\begin{itemize}
\item No direct way to measure TFP.
\bigskip
\item Solution: Infer growth in TFP as a residual. All the growth in output \textbf{NOT} accounted for by growth in inputs.\ldots
\begin{eqnarray*}
A = \frac{Y }{K^{\alpha}L^{1-\alpha}}
\end{eqnarray*}
\bigskip
\item Often called the ``Solow residual'' after Robert Solow who first showed how to measure it.
\end{itemize}
\end{frame}

%%%%%%%%%%%%%%%%%%%%%%%%%%%%%%%%%%%%%%%%%%%%%%%%%%%%%%%%%%%%%%%%%%%%%%%%%%%%%%%%%%%%%%%%%%%%%%%%
%%%%%%%%%%%%%%%%%%%%%%%%%%%%%%%%%%%%%%%%%%%%%%%%%%%%%%%%%%%%%%%%%%%%%%%%%%%%%%%%%%%%%%%%%%%%%%%%

\begin{frame}[t]
\frametitle{Step \#3: Log Difference the Production Function}
\begin{itemize}
\item From the production function, log difference everything.
\medskip
\item And from our previous steps we can measure each component.
\begin{eqnarray*}
\Delta \log Y = & \underbrace{\Delta \log A}_{\mbox{Change in Tehcnology}} \dots \\
\\
& +  \underbrace{\alpha \times (\Delta \log K)}_{\mbox{Change in Capital}} \dots \\
\\
\vspace{1cm}
& + \underbrace{(1-\alpha)\times(\Delta \log L)}_{\mbox{Change in Labor}}
\end{eqnarray*}
\end{itemize}
\end{frame}

%%%%%%%%%%%%%%%%%%%%%%%%%%%%%%%%%%%%%%%%%%%%%%%%%%%%%%%%%%%%%%%%%%%%%%%%%%%%%%%%%%%%%%%%%%%%%%%%
%%%%%%%%%%%%%%%%%%%%%%%%%%%%%%%%%%%%%%%%%%%%%%%%%%%%%%%%%%%%%%%%%%%%%%%%%%%%%%%%%%%%%%%%%%%%%%%%

\begin{frame}[t]
\frametitle{Growth in Living Standards in the US\ldots}
\begin{itemize}
\item Since the recession, real wages ($W/P$) have grown very little? Why? How would you use these tools to answer this question.
\bigskip
\item Prof. Robert Gordon's new book ``The Rise and Fall of American Growth: The U.S. Standard of Living Since the Civil War''
\begin{itemize}
\medskip
\item Argues that technology will not advance as it has in the past.
\medskip
\item This implies slower growth in living standards. Why?
\medskip
\item What do you think?
\end{itemize}
\end{itemize}
\end{frame}

%%%%%%%%%%%%%%%%%%%%%%%%%%%%%%%%%%%%%%%%%%%%%%%%%%%%%%%%%%%%%%%%%%%%%%%%%%%%%%%%%%%%%%%%%%%%%%%%
%%%%%%%%%%%%%%%%%%%%%%%%%%%%%%%%%%%%%%%%%%%%%%%%%%%%%%%%%%%%%%%%%%%%%%%%%%%%%%%%%%%%%%%%%%%%%%%%

\begin{frame}[t]
\frametitle{Growth in Living Standards}
\begin{itemize}
\item Prior results were all about growth in total GDP. Can do the same thing on per worker or per worker basis\ldots
\medskip
\begin{eqnarray*}
\Delta \log \frac{Y}{L} = & \underbrace{\Delta \log A}_{\mbox{Change in Tehcnology}} \dots \\
\\
& +  \underbrace{\alpha \times \left(\Delta \log \frac{K}{L} \right)}_{\mbox{Change in Capital per Worker}}\\
\end{eqnarray*}
\bigskip
\item Ok, how would you connect this with standard of living?
\end{itemize}
\end{frame}

%%%%%%%%%%%%%%%%%%%%%%%%%%%%%%%%%%%%%%%%%%%%%%%%%%%%%%%%%%%%%%%%%%%%%%%%%%%%%%%%%%%%%%%%%%%%%%%%
%%%%%%%%%%%%%%%%%%%%%%%%%%%%%%%%%%%%%%%%%%%%%%%%%%%%%%%%%%%%%%%%%%%%%%%%%%%%%%%%%%%%%%%%%%%%%%%%

\begin{frame}[t]
\frametitle{Problem Set \#2: Growth in China}
\begin{itemize}
\item Problem Set \#2. You do this for China.
\end{itemize}
\end{frame}


%%%%%%%%%%%%%%%%%%%%%%%%%%%%%%%%%%%%%%%%%%%%%%%%%%%%%%%%%%%%%%%%%%%%%%%%%%%%%%%%%%%%%%%%%%%%%%%%
%%%%%%%%%%%%%%%%%%%%%%%%%%%%%%%%%%%%%%%%%%%%%%%%%%%%%%%%%%%%%%%%%%%%%%%%%%%%%%%%%%%%%%%%%%%%%%%%

\begin{frame}[t]
\frametitle{Important Questions to Think About\ldots}
\begin{itemize}
\item If most of growth is projected to come from capital, is this good or worrying? Why or why not?
\bigskip
\item If most of growth is projected to come from TFP, is this good or worrying? Why or why not?
\bigskip
\item Can you connect your answers to core principles about the production function?
\end{itemize}
\end{frame}

%%%%%%%%%%%%%%%%%%%%%%%%%%%%%%%%%%%%%%%%%%%%%%%%%%%%%%%%%%%%%%%%%%%%%%%%%%%%%%%%%%%%%%%%%%%%%%%%%%
%%%%%%%%%%%%%%%%%%%%%%%%%%%%%%%%%%%%%%%%%%%%%%%%%%%%%%%%%%%%%%%%%%%%%%%%%%%%%%%%%%%%%%%%%%%%%%%%%%


\end{document} 